\section{Zufallsvariablen}
$X:\Omega\to\R$ ist \textbf{reellwertige Zufallsvariable}, wenn
\[\forall I \text{ Intervall}: \{\omega\in\Omega\mid X(\omega)\in I\}\in\A.\]
$X:\Omega\to\R^d$ ist \textbf{reellwertiger Zufallsvektor}, wenn jede Komponente Zufallsvariable ist.\\
\ \newline
Nützlich hierbei:\\
$\forall x\in\R:\{X\leq x\}\in\A
\Leftrightarrow\forall I \text{ Intervall}:\{X\in I\}\in\A$\\
\ \newline
\textbf{Verteilungsfunktion}: $\F_X:\R\to[0,1],\ x\mapsto\P(\{X\leq x\})$\\
ist monoton wachsend, rechtsseitig stetig,$\lim\limits_{x\to\infty}\P(x)=1$,
$\lim\limits_{x\to -\infty}\P(x)=0$, 
\mbox{$\F_X(x)-\F_X(x\textbf{-})=\P(\{X=x\})$}\\

\textbf{Verteilung}: $\P_X: \B{d}\to [0,1],\ B\mapsto\P(\{X\in B\})$\\
ist W'Maß auf $\B{d}$\\
\ \newline\newline

\begin{table}[h]
\centering
\caption*{\textbf{Vergleich}}
\begin{tabular}{P{0.45\linewidth} | P{0.45\linewidth}}
\textbf{diskret} & \textbf{absolut stetig} \\
\multicolumn{2}{c}{\textit{Definition}}  \\
$X$ diskret, wenn $\P(\{X\in D\})=1$ für abzählbares $D\subset\R$ (kleinstes solches $D=:D_X$ heißt Träger von $X$). & \\
Dann ex. Wahrscheinlichkeitsfunktion:
\[f_X:D_X\to [0,1],\ x\mapsto \P(\{X=x\})\] &        
$X$ absolut stetig, wenn $\P_X$ Dichte $f_X$ besitzt\\
\multicolumn{2}{c}{\textit{Verteilungsfunktion}}  \\
$\F_X(x)=\sum\limits_{y\in D_X\cap (-\infty, x]} \P(\{X=y\})$ &
$\F_X(x)=\int_{(-\infty,x]}f_X(y) d\lambda(y)$ \\
\multicolumn{2}{c}{\textit{Verteilung}}  \\
$\P_X(A)=\sum\limits_{x\in D_X\cap A} \P(\{X=x\})$ &
$\P_X(A)=\int_{A} f_X(x) d\lambda(x)$ \\
        &               
\end{tabular}
\end{table}



