\section{Zufallsvariablen}
\begin{itemize}
\item \textbf{Reellwertige Zufallsvariable}\\
$X:\Omega\to\R$ ist \textit{reellwertige Zufallsvariable}, wenn
\[
	\forall I \text{ Intervall}: \{\omega\in\Omega\mid X(\omega)\in I\}\in\A.
\]
Wenn $\A=\mathfrak{P}(\Omega)$, dann ist jedes $X:\Omega\to\R$ Zufallsvariable.\\

\item \textbf{Verteilungsfunktion}\\
Die \textit{Verteilungsfunktion}
\[
	\F_X:\R\to[0,1],\ x\mapsto\P(\{X\leq x\})
\]
einer Zufallsvariablen ist monoton wachsend, rechtsseitig stetig und es gilt\\
$\lim\limits_{x\to\infty}\P(x)=1$,
$\lim\limits_{x\to -\infty}\P(x)=0$, 
\mbox{$\F_X(x)-\F_X(x\textbf{-})=\P(\{X=x\})$}.

\item \textbf{Reellwertiger Zufallsvektor}\\
$X=(X_1,\ldots,X_d):\Omega\to\R^d$ ist \textit{reellwertiger Zufallsvektor}, wenn jede
Komponente $X_{i\in\{1,\ldots,d\}}$ reellwertige Zufallsvariable ist. Es gilt:\\
\[
	X \text{ ist Zufallsvektor } \Leftrightarrow X \text{ ist Borel-messbar.}
\]

\item \textbf{Verteilung}\\
Die \textit{Verteilung} eines Zufallsvektors $X$: 
\[
	\P_X: \B{d}\to [0,1],\ B\mapsto\P(\{X\in B\})
\]
ist ein Wahrscheinlichkeitsmaß auf $\B{d}$.

\item \textbf{Randverteilung}\\
Zu einem Zufallsvektor $X=(X_1,\ldots,X_n)$ heißen ($i\in\{1,\ldots,d\}$)
\[
	\P_{X_i} :\B{1}\to[0,1],\ B\mapsto \P(\{X_i\in B\})
\]
die \textit{(eindimensionalen) Randverteilungen} von $X$.

\item \textbf{Wahrscheinlichkeitsfunktion}\\
Eine Funktion $f:\Omega\to[0,\infty)$ heißt \textit{Wahrscheinlichkeitsfunktion} falls
\[
	  \sum_{\omega\in\Omega} f(\omega)=1
\]
für ein abzählbares $\Omega\in\R$.

\item \textbf{Wahrscheinlichkeitsdichte}\\
Eine Funktion $f\in\mathcal{F}_d^+$ heißt \textit{Wahrscheinlichkeitsdichte} falls
\[
	 \int_{\R^d} f(x)~d\lambda(x)=1.
\]

\item
Wahrscheinlichkeitsfunktionen/-dichten definieren Wahrscheinlichkeitsmaß auf $\A$
durch 
\[
\P(A)=\sum_{\omega\in A}f(\omega) \text{ bzw. } \P(A)=\int_A f(x)~d\lambda(x)
\]
für $A\in\A$.

\item
Zu jeder Wahrscheinlichkeitsfunktion/-dichte $f$ gibt es $(\Omega,\A,\P)$ sodass
darauf eine (diskrete/absolut stetige) Zufallsvariable $X$ mit $f_X=f$ existiert,
daher definiert jedes Wahrscheinlichkeitsmaß $\P$ auf $\B{d}$ einen
$d$-dimensionalen Zufallsvektor $X$ mit $\P_X=X$.

\item \textbf{Diskrete Zufallsvariablen}\\
Eine Zufallsvariable heißt \textit{diskret}, wenn $\P(\{X\in D\})$ für ein abzählbares
$D\subset\R$ gilt. Das kleinste solche $D=:D_X$ heißt \textit{Träger} der
Zufallsvariablen $X$.\\
Es gilt: 
$\underbrace{(\Omega,\A,\P) \text{ diskret}}_{\text{also $\Omega$ abzählbar}}$
$\Rightarrow X(\Omega)$ abzählbar $\Rightarrow X$ diskret.

\item \textbf{Diskrete Zufallsvektoren}\\
Ein $d$-dimensionaler Zufallsvektor $X=(X_1,\ldots,X_d)$ heißt \textit{diskret},
wenn es ein abzählbares $D\subset\R^d$ mit $\P(\{X\in D\})=1$ gibt.\\
Dann ist $D_X=\{x\in\R^d\mid \P(\{X=x\})>0\}\neq D_{X_1}\times\cdots\times D_{X_d}$
der Träger von $X$.\\
Es gilt: $X$ diskret $\Leftrightarrow\forall i\in\{1,\ldots,d\}: X_i$ diskret.\\
Für $h:\R^d\to\R$ Borel-messbar ist $h(X)$ diskret mit $D_{h(X)}=h(D_X)$.

\item \textbf{Absolut stetige Zufallsvektoren}\\
Ein Zufallsvektor $X$ heißt \textit{absolut stetig verteilt} falls die Verteilung
$\P_X$ eine Dichte $f_X$ besitzt.

\item
\textbf{Unabhängigkeit von Zufallsvariablen}\\
Eine Folge $(X_i)_{i\in I}$ von Zufallsvariablen heißt \textit{unabhängig} wenn für
jede Folge $(J_i)_{i\in I}$ von Intervallen die Folge der Ereignisse
$(\{X_i\in J_i\})_{i\in I}$ unabhängig ist.\\
Wegen:\\
$\forall x\in\R:\{X\leq x\}\in\A
\Leftrightarrow\forall I \text{ Intervall}:\{X\in I\}\in\A$\\ \newline
ist $(X_i)_{i\in I}$ genau dann unabhängig, wenn für 
$\emptyset\neq\tilde{I}\subset I$, $(x_i)_{i\in\tilde{I}}\subset\R$ gilt:
\[
	\P(\bigcap_{i\in\tilde{I}}\{X_i\leq x_i\})=\prod_{i\in\tilde{I}}\{X_i\leq x_i\}.
\]
Die Unabhängigkeit einer Folge $(X_i)_{i\in I}$ impliziert die paarweise
Unabhängigkeit von $X_i,X_j$ für $i,j\in I$ mit $i\neq j$.

Beachte auch \hyperlink{unabhaengig}{dieses}.

Es gilt: $X,X$ unabhängig $\Leftrightarrow\exists c\in\R:\P(\{X=c\})=1$ \U

\item 
\textbf{Identische Verteilung}\\
Zwei Zufallsvariablen $X,Y$ heißen \textit{identisch verteilt}, wenn $F_X=F_Y$.\\
Zwei Zufallsvektoren $\tilde{X}=(\tilde{X}_1,\ldots,\tilde{X}_d),
\tilde{Y}=(\tilde{Y}_1,\ldots,\tilde{Y}_d)$ heißen \textit{identisch verteilt}, wenn
$P_{\tilde{X}}=P_{\tilde{Y}}$.\\
Es gilt $\tilde{X},\tilde{Y}$ identisch verteilt
$\Rightarrow\forall i\in\{1,\ldots,d\}: \tilde{X}_i,\tilde{Y}_i$ identisch verteilt

\end{itemize}

\newpage
\begin{table}[h]
\centering
\caption*{\textbf{Vergleich}}
\begin{tabular}{P{0.45\linewidth} | P{0.45\linewidth}}
\textbf{diskret} & \textbf{absolut stetig} \\
\multicolumn{2}{c}{\textit{W'funktion/Dichte}}  \\
Es existiert W'funktion: &\\
$f_X:D_X\to [0,1],\ x\mapsto \P(\{X=x\})$ &        
$\P_X$ besitzt Dichte $f_X$\\
&\\
\multicolumn{2}{c}{\textit{Verteilungsfunktion} ($d=1$)}  \\
$$\F_X(x)=\sum\limits_{y\in D_X\cap (-\infty, x]} \P(\{X=y\})$$ &
$$\F_X(x)=\int_{(-\infty,x]}f_X(y) d\lambda(y)$$ \\
\multicolumn{2}{c}{\textit{Verteilung}}  \\
$$\P_X(A)=\sum\limits_{x\in D_X\cap A} \P(\{X=x\})$$ &
$$\P_X(A)=\int_{A} f_X(x) d\lambda(x)$$ \\
\multicolumn{2}{c}{\textit{Randverteilung/Randdichte}}\\
\multicolumn{2}{c}{$X=(X_1\ldots,X_d)$}\\
todo       &
$f_{X_i}(x)=$
\mbox{$\int_\R\cdots\int_\R f_X(x_1,\ldots,x_{i-1},x,x_{i+1},\ldots,x_d)
d\lambda(x_1)\cdots d\lambda_{x_d}$}\\
&\\
&(alles integrieren außer $i$-te Koordinate)  \\
&\\
\multicolumn{2}{c}{\textit{$X_1,\ldots,X_d$ unabhängig}}\\
$\forall x_1\ldots,x_n\in\R:$
\mbox{$\P(\bigcap_{i=1}^n \{X_i=x_i\})=\prod_{i=1}^n\P(\{X_i=x_i\})$} &
$\forall B_1,\ldots,B_n\in\B{1}:$
\mbox{$\P(\bigcap_{i=1}^n \{X_i\in B_i\})=\prod_{i=1}^n\P(\{X_i\in B_i\})$} \\
&$\Leftrightarrow$\\ 
 &
$\forall B_1,\ldots,B_n\in\B{1}:$
\mbox{$\P(B_1\times\cdots\times B_d)=\prod_{i=1}^d\P_{X_i}(B_i)$}\\
&\\
& Gemeinsame Verteilung $P_X$ von $X_1,\ldots,X_d$ ist also Produkt der
Randverteilungen $P_{X_{i\in\{1,\ldots,d\}}}$\\
&\\
\multicolumn{2}{c}{\textit{$X,Y$ identisch verteilt} ($d=1$)}\\
$D_X=D_Y$ und $f_X=f_Y$ &
$P_X=P_Y$\\
$\forall z\in\R:\P(\{X=z\})=\P'(\{Y=z\})$&
$\forall A\in\B{1}:\P(\{X\in A\})=\P'(\{Y\in A\})$\\

\end{tabular}
\end{table}



