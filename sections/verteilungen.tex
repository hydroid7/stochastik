\section{Verteilungen}

\textbf{Bernoulli-Verteilung} $X\sim\mathbf{B}(1,p)$ mit $p\in [0,1]$
\begin{itemize}
\item $\P(\{X=1\})=p,\ \P(\{X=0\})=1-p$

\item diskret mit $D_X=\begin{cases}
\{0,1\} 	& \text{falls } p\in (0,1)	\\
\{0\}	& \text{falls } p=0			\\
\{1\} 	& \text{falls } p=1
\end{cases}$

\item $\F_X(x)=\begin{cases}
0 	& \text{für } x<0				\\
1-p	& \text{für } 0\leq x <1			\\
1 	& \text{für } x\geq 1
\end{cases}$

\item $\E(X)=p$, $\Var(X)=p-p^2=p(1-p)$

\end{itemize}

\textbf{Binomial-Verteilung} $X\sim\mathbf{B}(n,p)$ mit $p\in [0,1]$
\begin{itemize}
\item $\forall k\in\{0,\ldots,n\}:\P(\{X=k\})=\binom{n}{k}\cdot p^k\cdot (1-p)^{n-k}$

\item diskret mit
$D_X=\begin{cases}
\{0,\ldots,n\} 	& \text{falls } p\in (0,1)	\\
\{0\}			& \text{falls } p=0			\\
\{n\} 			& \text{falls } p=1
\end{cases}$

\item $\E(X)=n\cdot p$, $\Var(X)=n\cdot p\cdot (1-p)$

\item Anwendung: Zählen der Erfolge von $n$ unabhängigen, hintereinander ausgeführten
Experimenten mit Erfolgswahrscheinlichkeit $p_n$
\end{itemize}

\textbf{Hypergeometrische Verteilung} $X\sim\mathbf{H}(N,N_0,n)$
\begin{itemize}
\item $\forall l\in D_X:\P(\{X=l\})
=\frac{\binom{N_0}{l}\cdot\binom{N-N_0}{n-l}}{\binom{N}{n}}$

\item diskret mit 
$D_X=\{\operatorname{max}(0,n-(N-N_0)),\ldots,\operatorname{min}(N_0,n)\}$

\item Anwendung: unter $N$ Objekten finden sich $N_0$ markierte und es werden
$n$ entnommen
\end{itemize}

\textbf{Poisson-Verteilung} $X\sim\mathbf{P}(\lambda)$ für $\lambda>0$
\begin{itemize}
\item $\forall k\in\N_0:\P(\{X=k\})=\exp(-\lambda)\cdot\frac{\lambda^k}{k!}$

\item diskret mit $D_X=\N_0$

\item $X\sim\mathbf{P}(\lambda_2), Y\sim\mathbf{P}(\lambda_2)
\Rightarrow X+Y\sim\mathbf{P}(\lambda_2 + \lambda_2)$ \U

\item $\E(X)=\lambda=\Var(X)$, $\E(X^2)=\lambda^2+\lambda$

\item Anwendung: Approximation von $\mathbf{B}(n,p)$ durch $\mathbf{P}(\lambda)$
mit $\lambda=n\cdot p$ für 'große' $n$ und 'kleine' $p$, also Eintreffen eines
seltenen Ereignis bei großer Anzahl an Wiederholungen
\end{itemize}

\newpage
\textbf{Geometrische Verteilung} $X\sim\mathbf{G}(p)$ mit $p\in\textbf{(}0,1]$
\begin{itemize}
\item $\forall k\in\N:\P(\{X=k\})=p\cdot(1-p)^{k-1}$

\item diskret mit $D_X=\N$

\item $\E(X)=\frac{1}{p}$, $\Var(X)=\frac{1-p}{p^2}$, $\E(X^2)=\frac{2-p}{p^2}$

\item Gedächtnislos:\\ $\forall k_1,k_2\in\N \text{ mit } k_1<k_2:
\P(\{X>k_2\}\mid \{ X>k_1\}) = \P(\{X>k_2-k_1\})$

\item Anwendung: diskretes Warten bis zum ersten Eintritt eines Ereignisses
\end{itemize}

\textbf{Gleichverteilung} $X\sim\mathbf{U}([a,b])$ für $-\infty<a<b<\infty$
\begin{itemize}
\item absolut stetig mit 
$f_X(x)=\begin{cases}
\frac{1}{b-a} 	& \text{falls } x\in [a,b]	\\
0				& \text{sonst }		
\end{cases}$

\item
$\F_X(x)=\begin{cases}
0 				& \text{falls } x<a			\\
\frac{x-a}{b-a}	& \text{falls } x\in [a,b]	\\
1				& \text{sonst}
\end{cases}$

\item $\E(X)=\frac{a+b}{2}$, $\Var(X)=\frac{(b-a)^2}{12}$,
$\E(X^2)=\frac{b^3-a^3}{3(b-a)}$
\end{itemize}

\textbf{Einpunktverteilung} $X\sim\mathbf{U}(\{c\})$ für ein $c\in\R$
\begin{itemize}
\item $P(\{X=c\})=c$

\item diskret mit $D_X=\{c\}$

\item $\F_X(x)=\begin{cases}
0 	&\text{falls } x<c			\\
1	& \text{falls } x\geq c
\end{cases}$

\item $\E(X)=c$, $\Var(X)=0$
\end{itemize}

\textbf{Exponentialverteilung} $X\sim\mathbf{Exp}(\lambda)$ für $\lambda>0$
\begin{itemize}
\item absolut stetig mit
$f_X:\R\to [0,\infty],\ x\mapsto\begin{cases}
\lambda\cdot\exp(-\lambda x) 	& \text{falls } x\geq 0	\\
0							& \text{sonst }		
\end{cases}$

\item $\F_X(x)=1-\exp(-\lambda x)$ für $x>0$; $F_X(x)=0$ für $x\leq 0.$

\item $\E(X)=\frac{1}{\lambda}$, $\Var(X)=\frac{1}{\lambda^2}$,
$\E(X^2)=\frac{2}{\lambda^2}$

\item Gedächtnislos: $\forall s,t>0:\P(\{X>t+s\}\mid\{X>t\})=\P(\{X>s\})$

\item Anwendung: Warten bis zum ersten Eintritt eines Ereignisses (zB Lebensdauer,
radioaktiver Zerfall,\ldots)
\end{itemize}

\textbf{Standard-Normalverteilung} $X\sim\mathbf{N}(0,1)$
\begin{itemize}
\item absolut stetig mit $f_X:\R\to[0,\infty],\
x\mapsto\frac{1}{\sqrt{2\pi}}\cdot\exp(-\frac{x^2}{2})$

\item $\Phi(x):=\F_X(x)=
\frac{1}{\sqrt{2\pi}}\int_{-\infty}^{x}\exp(-\frac{y^2}{2})d\lambda(y)$
(keine explizite Formel)

\item $\forall z\in\R:\Phi(z)+\Phi(-z)=1$ \U

\item $\E(X)=0$, $\Var(X)=1$

\end{itemize}

\textbf{Normalverteilung} $X\sim\mathbf{N}(\mu,\sigma^2)$ für 
$\mu\in\R,\ \sigma>0$
\begin{itemize}
\item absstetig mit $f_X:\R\to[0,\infty],\ 
x\mapsto\frac{1}{\sqrt{2\pi\sigma^2}}\cdot\exp(-\frac{(x-\mu)^2)}{2\sigma^2})$

\item $X\sim\mathbf{N}(\mu,\sigma^2)
\Rightarrow a\cdot X+b\sim\mathbf{N}(a\cdot\mu+b,a^2\cdot\sigma^2)$

\item Also: $\frac{X-\mu}{\sigma}\sim\mathbf{N}(0,1)$

\item $\E(X)=\mu$, $\Var(X)=\sigma^2$
\end{itemize}

\textbf{Gleichverteilung} (mehrdimensional) $X\sim\mathbf{U}(S)$ für $S\subset\R^n$
\underline{endlich}
\begin{itemize}
\item $\forall x\in S: \P(\{X=x\})=\frac{1}{|S|}$

\item diskret mit $D_X=S$
\end{itemize}

\textbf{Gleichverteilung} (mehrdimensional) $X\sim\mathbf{U}(G)$ für $G\in\B{d}$
\begin{itemize}
\item $\P_X(A)=\frac{\lambda_d(A\cap G)}{\lambda_d(G)}=
\int_A \frac{1}{\lambda_d(G)\cdot 1_G(x)}d\lambda(x)$

\item absolut stetig mit $f_X=\frac{1}{\lambda_d(G)}\cdot \1_G$
\end{itemize}

\textbf{Standard-Normalverteilung} (mehrdimensional)
\begin{itemize}
\item absolut stetig mit 
$f_X:\R^d\to [0,\infty),\ 
x\mapsto(2\pi)^{-d/2}\cdot\exp(-\frac{1}{2}\sum_{i=1}^{d}x_i^2)$

\item $X=(X_1,\ldots,X_d)$ ist standard-normalverteilt 
$\Leftrightarrow$ $X_1,\ldots,X_d\ \operatorname{i.i.d}\sim\mathbf{N}(0,1)$
\end{itemize}
