\section{Integration}

\textbf{Lesbegue-Maß} $\lambda_d: \B{d}\to [0,\infty]$
\begin{itemize}
\item Für $B_1,B_2,\ldots\in\B{d}$ \underline{p.d.}:
$\lambda_d(\bigcup_{i=1}^{\infty} B_i)=\sum_{i=1}^{\infty}\lambda_d(B_i)$.

\item Für $a_1,b_1,\ldots,a_n,b_n\in\R$:
$\lambda_d([a_1,b_1]\times\cdots\times[a_n,b_n])=\prod_{i=1}^{d}(b_i-a_i)$.

\item Für $a\in\R^d$, $Q\in\mathrm{O}(d)$, $A\in\B{d}$: 
$\lambda_d(a+Q(A))=\lambda_d(A)$

\item $\forall x\in\R^d: \lambda_d(\{x\})=0$

\item $\lambda_d$ ist $\sigma$-subadditiv, $\sigma$-stetig von unten
(nicht von oben!), monoton
\end{itemize}

\textbf{Borel-messbare Funktionen}
\begin{itemize}
\item $f:\R^d\to\R$ \textit{Borel-messbar}
$:\Leftrightarrow\forall B\in\B{1}: \{f\in B\}\in\B{d}$

\item $f:\R^d\to\overline{\R}$ \textit{Borel-messbar}
$:\Leftrightarrow\forall B\in\B{1}: \{f\in B\}\in\B{d}$ und $\{f=\infty\}\in\B{d}$

\item $\mathcal{F}_d:=\{f:\R^d\to\overline{\R}\mid f \text{ ist Borel-messbar}\}$,
$\mathcal{F}_d^+ :=\{f\in\mathcal{F}_d\mid f\geq 0\}$

\item $\forall B\in\B{d}: \1_B\in\mathcal{F}_d^+$

\item $f$ stetig $\Rightarrow\ f\in\mathcal{F}_d$

\item $f,g\in\mathcal{F}_d\Rightarrow a\cdot f+g, f\cdot g, f/g,
\operatorname{max}(f,g)\in\mathcal{F}_d$ falls wohldefiniert ($a\in\overline{\R}$)
\end{itemize}

\textbf{Elementare Funktionen}
\begin{itemize}

\item Abbildung $e:\R^d\to\overline{\R}$ \textit{elementar} $:\Leftrightarrow$\\
$\exists b_1,\ldots,b_n\in\overline{\R}\wedge\exists B_1,\ldots,B_n\in\B{d}$
\underline{p.d.} mit $e(x)=\sum_{i=1}^{n} b_i\cdot \1_{B_i}(x)$ für alle $x\in\R^d$

\item $\mathcal{E}_d:=\{f:\R^d\to\overline{\R}\mid f \text{ ist elementar}\}$,
$\mathcal{E}_d^+:=\{f\in\mathcal{E}_d\mid f\geq 0\}$

\item $\mathcal{E}_d\subset\mathcal{F}_d$
\end{itemize}

\textbf{Lesbegue-Integral}
\begin{itemize}
\item Für $f\in\mathcal{E}_d^+$: 
$\int f~d\lambda_d:=\sum_{i=1}^{n} b_i\cdot\lambda_d(B_i)\in [0,\infty]$

\item Definiert für alle $f\in\mathcal{F}_d^+$

\item $\int_B f~\lambda_d:=\int (f\cdot \1_B)d\lambda_d$ für $B\in\B{d}$

\item Lesbegue-Integral ist monoton und linear

\item Für $f:\R\to\R$ Borell-messbar und beschränkt auf $[a,b]$ mit
\mbox{$\lambda(\{x\in[a,b]\mid f \text{ unstetig in }x\})=0$} gilt:
\[
	\underbrace{\int_{a}^{b}f(x)dx}_{\text{Riemann-Integral}}
	= \int_{[a,b]} f~d\lambda
\]

\item $\forall A\in\B{d}:\int_A f~d\lambda_d = \int_A g~d\lambda_d$
$\Leftrightarrow \lambda_d(\{x\in\R^d\mid f(x)\neq g(x)\})=0$
\end{itemize}

\textbf{Satz von der monotonen Konvergenz}\\
Für jede monoton wachsende Folge $0\leq f_1\leq f_2\leq\ldots$
in $\mathcal{F}_d^+$ gilt
\[\lim_{n\to\infty}\int f_n~d\lambda_d=
\int\lim_{n\to\infty} f_n~d\lambda_d\in [0,\infty].\]

\textbf{Satz von der dominierten Konvergenz}\\
Für $f,f_1,f_2,\ldots\in\mathcal{F}_d$ mit $\lim_{n\to\infty} f_n=f$ 
und $g\in\mathcal{F}_d$ integrierbar mit $|f_n|\leq g$ für alle $n\in\N$ gilt
\[
	\int fd\lambda_d=\lim_{n\to\infty} \int f_n~d\lambda_d.
\]
Insbesondere für $f\in\mathcal{F}_1$ integrierbar oder $f\in\mathcal{F}_1^+$
\[
	\int_\R f~d\lambda=\lim_{n\to\infty} \int_{[n,n]} f~d\lambda
\]


\textbf{Satz von Fubini}\\
Für $f\in\mathcal{F}_d$ integrierbar oder $f\in\mathcal{F}_d^+$ und
$B = B_1\times\cdots\times B_d\subset\B{1}^d$ gilt
\[
	\int_B f~d\lambda_d=\int_{B_{i_1}}\cdots\int_{B_{i_d}} f(x_1,\ldots,x_d)
	d\lambda(x_{i_d})\cdots d\lambda(x_{i_1})
\]
für jede Permutation $(i_1,\ldots,i_d)$ von $(1,\ldots,d)$.

\textbf{Substitutionsregel} (Riemann-Integration)\\
Für $a,b\in\R$ mit $a<b$, $g:[a,b]\to I$ stetig diferenzierbar, $f:I\to\R$ stetig gilt
\[
	\int_a^b f(g(y))\cdot g'(y)dy=\int_{g(a)}^{g(b)}f(x)dx.
\]

\textbf{Partielle Integration} (Riemann-Integration)\\
Für $a,b\in\R$ mit $a<b$ und $f,g:[a,b]\to\R$ stetig differenzierbar gilt
\[
	\int_a^b f'(x)\cdot g(x)dx 
	= f(x)\cdot g(x)\Big\vert_a^b-\int_a^b f(x)\cdot g'(x)dx.
\]
