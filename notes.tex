\documentclass[11pt,a4paper,ngerman]{article}

% IMPORTS
\usepackage[ngerman]{babel}
\usepackage[utf8x]{inputenc}
\usepackage[colorinlistoftodos,prependcaption,textsize=tiny]{todonotes} % To-Dos
\usepackage{parskip} 	% Absätze ohne Einrückungen
\usepackage{mathtools} 	% Für Pfeil mit Erklärung
\usepackage{amsfonts} 	% Buchstaben mit Doppelstrich
\usepackage{amsmath}
\usepackage{amssymb}
\usepackage{amsthm}		
\usepackage{bbm}			% Indikator Funktion
\usepackage{verbatim}	% comment-env
\usepackage{wasysym}		% \lightning
\usepackage{graphicx}	% Einbinden von Grafiken
\usepackage[left=3cm,right=3cm,top=3cm,bottom=3.5cm,includeheadfoot]{geometry}
\usepackage{url}
\usepackage{hyperref}

% CENTERED COLUMNS WITH FIX WIDTH
\usepackage{array}
\newcolumntype{P}[1]{>{\centering\arraybackslash}p{#1}}

% NEWCOMMANDS
\newcommand{\A}{\mathcal{A}}
\newcommand{\B}[1]{\mathcal{B}_{#1}}
\renewcommand{\P}{\operatorname{P}}
\newcommand{\F}{\operatorname{F}}
\newcommand{\Var}{\operatorname{Var}}
\newcommand{\E}{\operatorname{E}}
\newcommand{\R}{\mathbb{R}}
\newcommand{\N}{\mathbb{N}}
\newcommand{\1}{\mathbbm{1}}
\newcommand{\compl}[1]{{#1}^{\text{\tiny C}}}
\newcommand{\U}{\texttt{(Übung)}}

% META
\date{07.02.2018}
\author{Maximilian Reif}
\title{Notizen zur Einführung in die Stochastik}

%--------------------------------------------------------------------------------------------------------------

\begin{document}

\begin{titlepage}
    \ \newline\newline\newline\newline\newline
	
	\begin{center}

		\huge Notizen zu\\
		\Huge\textbf{Einführung in die Stochastik}\\
		\huge bei Prof. Kaiser im WS 17/18\\
		\normalsize

		\vspace{1cm}
		\begin{tabular}[b]{l|l}
			\textbf{author} 		& Maximilian Reif
			\texttt{\href{mailto:reifmaxi@fim.uni-passau.de}
			{<reifmaxi@fim.uni-passau.de>}}\\
			\textbf{last change}	& 8. Feb 2018 \\
			\textbf{version} 	& 0.6.2\\
			\textbf{github} 		& \url{https://github.com/lordreif/stochastik}
		\end{tabular}
		\vspace{1cm}
		
	\end{center}
	
	\begin{figure}[b]
	\centering
	\includegraphics[width=0.6\textwidth]{img/fim.png}
	\end{figure}
	
\end{titlepage}

%\section*{Disclaimer}
\thispagestyle{empty}

Dies ist eine Kurzzusammenfassung des Stoffes der Vorlesung
\textit{Einführung in die Stochastik} im Wintersemester 17/18 an der Universität Passau.

Das Dokument erhebt nicht den Anspruch, in sich geschlossen zu sein, und verzichtet
gänzlich auf Beweise.\\
Da die Verständlichkeit über Formales gestellt wurde, besteht weder auf mathematische
Richtigkeit noch auf Vollständigkeit Anspruch.


Falls Sie Fehler in Layout und Sprache finden, oder sich falsche Behauptungen
eingeschlichen haben, teilen Sie dies bitte dem Autor mit.


\newpage
\tableofcontents\thispagestyle{empty}
\newpage

\setcounter{page}{1}
\section{Basics}

\begin{itemize}
\item \mbox{\textit{Ergebnis:} Element aus $\Omega$, 
\textit{Ereignis: } Element aus $\A$ (Teilmenge von $\Omega$)}

\item \textit{$\sigma$-Algebra $\A$} $\subset\mathfrak{P}(\Omega)$
	\begin{enumerate}
	\item $\Omega\in\A$
	\item $A\in\A \Rightarrow \compl{A}\in\A$
	\item $A_1, A_2,\ldots\in\A \Rightarrow \bigcup_{i=1}^{\infty} A_i \in\A$
	\end{enumerate}
	
\item \textit{Wahrscheinlichkeitsmaß:} $\P:\A\rightarrow [0,1]$
	\begin{enumerate}
	\item $\P(\Omega)=1$
	\item $A_1,A_2,\ldots\in\A$ \underline{p.d.} 
	$\Rightarrow \P(\bigcup_{i=1}^{\infty} A_i)
	= \sum_{i=1}^{\infty}\P(A_i)$ \hfill\textit{($\sigma$-Additivität)}
	\end{enumerate}

\item $A,B\in\A:$
	\begin{enumerate}
	\item $A\subset B \Rightarrow \P(B)=\P(A)+\P(B\backslash A)$
	\item $A\subset B \Rightarrow \P(A)\leq\P(B)$
	\hfill\textit{(Monotonie)}
	\item $\P(A^{\text{C}})=1-\P(A)$
	\item $\P(A\cup B) = \P(A)+\P(B)-\P(A\cap B)$
	\end{enumerate}
	
\item $A_1,A_2,\ldots\in\A$
	\begin{enumerate}
	\item $\P(\bigcup_{i=1}^{\infty} A_i)\leq\sum_{i=1}^{\infty} \P(A_i)$ 
	\hfill\textit{($\sigma$-Subadditivität)}
	\item $A_1\subset A_2\subset\ldots \Rightarrow 
	\P(\bigcup_{i=1}^{\infty} A_i)=\lim\limits_{i\to\infty} \P(A_i)$
	\hfill\mbox{(\textit{$\sigma$-Stetigkeit von unten})}
	\item $A_1 \supset A_2\supset\ldots \Rightarrow
	\P(\bigcap_{i=1}^{\infty} A_i)=\lim\limits_{i\to\infty} \P(A_i)$
	\hfill\mbox{(\textit{$\sigma$-Stetigkeit von oben})}
	\end{enumerate}

\item $\P(B)>0$, dann $\P(\ \cdot\mid B) := \frac{\P(\ \cdot\ \cap B)}{\P(B)}$
W'Maß auf $\A$ mit $\P(B\mid B)=1$

\item \textbf{Formel der totalen Wahrscheinlichkeit}\\
Für $B_1, B_2,\ldots,B_n\in\A$ \underline{p.d.} mit $\P(B_i)>0$ für alle
$i\in\{1,\ldots,n\}$ und $\bigcup_{i=1}^{n}B_i=\Omega$ gilt:
\[
	\P(A)=\sum_{i=1}^{n} \P(A\mid B_i)\cdot\P(B_i)\text{ für alle } A\in\A
\]

\item \textbf{Formel von Bayes}\\
Für $B_1, B_2,\ldots,B_n\in\A$ \underline{p.d.} mit $\P(B_i)>0$ für alle
$i\in\{1,\ldots,n\}$ und $\bigcup_{i=1}^{n}B_i=\Omega$ gilt:
\[
	\P(B_i\mid A)=
	\frac{\P(A\mid B_i)\cdot\P(B_i)}{\sum\limits_{j=1}^{n} \P(A\mid B_j)\cdot\P(B_j)}\ 
	\text{für alle } A\in\A \text{ mit } \P(A)>0
\]

\item $\P(B\mid A)=\frac{\P(A\mid B)\cdot\P(B)}{\P(A)}$ für $A,B\in\A$ mit
$\P(A),\P(B)>0$

\item $\P(A\cap B)=\P(A\mid B)\cdot\P(B)$ für $B\in\A$ mit $\P(B)>0$

\item $A_1,\ldots ,A_n\in\A$ mit $\P(\bigcap_{i=1}^{n-1} A_i)>0$, dann\\
$\P(\bigcap_{i=1}^{n} A_i)=$ 
$P(A_1)\cdot\P(A_2\mid A_1)\cdots
\P(A_1\cap\ldots\cap A_n)\cdot\P(A_n\mid A_1\cap\ldots\cap A_{n-1})$

\item $A,B$ unabhängig $:\Leftrightarrow \P(A\cap B)=\P(A)\cdot\P(B)
\Leftrightarrow P(A\mid B) = P(A)$ falls $\P(B)>0$

\item $A_1,\ldots ,A_n$ unabhängig 
$\Leftrightarrow\P(\bigcap_{i=1}^{n}A_i)=\prod_{i=1}^{n} P(A_i)
\Rightarrow$ paarweise Unabhängigkeit

\item $\binom{n}{k}:= \frac{n!}{k!(n-k)!}$, 
$\binom{n}{k-1}+\binom{n}{k}=\binom{n+1}{k}$,
$\binom{n}{k}=\binom{n}{n-k}$

\item $ |N_1 \times N_2 \times\dots\times N_k| = |N_1| \cdot |N_2| \cdots |N_k|$,
speziell $|N^k| = {|N|}^k$

\item \mbox{$n=|N|,k\leq n$, dann $|\{(\omega_1,\ldots,\omega_k)\in N^k 
\mid\omega_i\neq\omega_j\text { für }i\neq j\}|=\frac{n!}{(n-k)!}$}

\item $n=|N|,k\leq n$, dann $|\{K\subset N\mid |K|=k\}| = \binom{n}{k} =$
\mbox{$|\{(\omega_1,\ldots,\omega_k)\in N^k \mid 1\leq\omega_1<\ldots<\omega_k\leq n\}|$}

\item $k,n\in\mathbb{N}$, dann 
$|\{(\omega_1,\ldots,\omega_k)\in N^k\mid 1\leq\omega_1\leq\ldots\leq\omega_k\leq n\}|
=\binom{n+k-1}{k}$

\item Insgesamt:

\begin{table}[h]
\centering
\begin{tabular}{l|l|l|}
\cline{2-3}
                                       & \begin{tabular}[c]{@{}l@{}} mit Berücksichtigung\\ der Reihenfolge\end{tabular} & \begin{tabular}[c]{@{}l@{}}ohne Berücksichtigung\\ der Reihenfolge\end{tabular}                                                                   \\ \hline
\multicolumn{1}{|l|}{mit Zurücklegen}  & \begin{tabular}[c]{@{}l@{}}$\Omega=N^k$,\\ $|\Omega| = n^k$\end{tabular}   & \begin{tabular}[c]{@{}l@{}}$\Omega = \{\omega\in N^k\mid1\leq\omega_1\leq\ldots\leq\omega_n\leq n\}$,\\ $|\Omega| = \binom{n+k-1}{k}$\end{tabular} \\ \hline
\multicolumn{1}{|l|}{ohne Zurücklegen} & \begin{tabular}[c]{@{}l@{}}$\Omega=\{\omega\in N^k \mid \omega_i\neq\omega_j \text{ für } i\neq j\} $\\ $|\Omega|=\frac{n!}{(n-k)!}$\end{tabular}  & \begin{tabular}[c]{@{}l@{}}$\Omega=\{\omega\in N^k\mid 1\leq\omega_1<\ldots<\omega_k\leq n\}$,\\ $|\Omega|=\binom{n}{k}$\end{tabular}              \\ \hline
\end{tabular}
\end{table}

\end{itemize}
\section{Integration}

\textbf{Lesbegue-Maß} $\lambda_d: \B{d}\to [0,\infty]$
\begin{itemize}
\item Für $B_1,B_2,\ldots\in\B{d}$ \underline{p.d.}:
$\lambda_d(\bigcup_{i=1}^{\infty} B_i)=\sum_{i=1}^{\infty}\lambda_d(B_i)$.

\item Für $a_1,b_1,\ldots,a_n,b_n\in\R$:
$\lambda_d([a_1,b_1]\times\cdots\times[a_n,b_n])=\prod_{i=1}^{d}(b_i-a_i)$.

\item Für $a\in\R$, $Q\in\mathrm{O}(d)$, $A\in\B{d}$: $\lambda_d(a+Q(A))=\lambda_d(A)$

\item $\forall x\in\R^d: \lambda_d(\{x\})=0$

\item $\lambda_d$ ist $\sigma$-subadditiv, $\sigma$-stetig von unten
(nicht von oben!), monoton
\end{itemize}

\textbf{Borel-messbare Funktionen}
\begin{itemize}
\item $f:\R^d\to\R$ \textit{Borel-messbar}
$:\Leftrightarrow\forall B\in\B{1}: \{f\in B\}\in\B{d}$

\item $f:\R^d\to\overline{\R}$ \textit{Borel-messbar}
$:\Leftrightarrow\forall B\in\B{1}: \{f\in B\}\in\B{d}$ und $\{f=\infty\}\in\B{d}$

\item $\mathcal{F}_d:=\{f:\R^d\to\overline{\R}\mid f \text{ ist Borel-messbar}\}$,
$\mathcal{F}_d^+ :=\{f\in\mathcal{F}_d\mid f\geq 0\}$

\item $\forall B\in\B{d}: \1_B\in\mathcal{F}_d^+$

\item $f$ stetig $\Rightarrow\ f\in\mathcal{F}_d$

\item $f,g\in\mathcal{F}_d\Rightarrow a\cdot f+g, f\cdot g, f/g,
\operatorname{max}(f,g)\in\mathcal{F}_d$ falls wohldefiniert ($a\in\overline{\R}$)
\end{itemize}

\textbf{Elementare Funktionen}
\begin{itemize}

\item Abbildung $e:\R^d\to\overline{\R}$ \textit{elementar} $:\Leftrightarrow$\\
$\exists b_1,\ldots,b_n\in\overline{\R}\wedge\exists B_1,\ldots,B_n\in\B{d}$
\underline{p.d.} mit $e(x)=\sum_{i=1}^{n} b_i\cdot \1_{B_i}$ für alle $x\in\R^d$

\item $\mathcal{E}_d:=\{f:\R^d\to\overline{\R}\mid f \text{ ist elementar}\}$,
$\mathcal{E}_d^+:=\{f\in\mathcal{E}_d\mid f\geq 0\}$

\item $\mathcal{E}_d\subset\mathcal{F}_d$
\end{itemize}

\textbf{Lesbegue-Integral}
\begin{itemize}
\item Für $f\in\mathcal{E}_d^+$: 
$\int f~d\lambda_d:=\sum_{i=1}^{n} b_i\cdot\lambda_d(B_i)\in [0,\infty]$

\item Definiert für alle $f\in\mathcal{F}_d^+$

\item $\int_B f~\lambda_d:=\int (f\cdot \1_B)d\lambda_d$ für $B\in\B{d}$

\item Lesbegue-Integral ist monoton und linear

\item Für $f:\R\to\R$ Borell-messbar und beschränkt auf $[a,b]$ mit
\mbox{$\lambda(\{x\in[a,b]\mid f \text{ unstetig in }x\})=0$} gilt:
\[
	\underbrace{\int_{a}^{b}f(x)dx}_{\text{Riemann-Integral}}
	= \int_{[a,b]} f~d\lambda
\]

\item $\forall A\in\B{d}:\int_A f~d\lambda_d = \int_A g~d\lambda_d$
$\Leftrightarrow \lambda_d(\{x\in\R^d\mid f(x)\neq g(x)\})=0$
\end{itemize}

\textbf{Satz von der monotonen Konvergenz}\\
Für jede monoton wachsende Folge $0\leq f_1\leq f_2\leq\ldots$
in $\mathcal{F}_d^+$ gilt
\[\lim_{n\to\infty}\int f_n~d\lambda_d=
\int\lim_{n\to\infty} f_n~d\lambda_n\in [0,\infty].\]

\textbf{Satz von der dominierten Konvergenz}\\
Für $f,f_1,f_2,\ldots\in\mathcal{F}_d$ mit $\lim_{n\to\infty} f_n=f$ 
und $g\in\mathcal{F}_d$ integrierbar mit $|f_n|\leq g$ für alle $n\in\N$ gilt
\[
	\int fd\lambda_d=\lim_{n\to\inf} \int f_n~d\lambda_d.
\]
Insbesondere für $f\in\mathcal{F}_1$ integrierbar oder $f\in\mathcal{F}_1^+$
\[
	\int_\R f~d\lambda=\lim_{n\to\infty} \int_{[n,n]} f~d\lambda
\]


\textbf{Satz von Fubini}\\
Für $f\in\mathcal{F}_d$ integrierbar oder $f\in\mathcal{F}_d^+$ und
$B = B_1\times\cdots\times B_n\subset\B{1}^d$ gilt
\[
	\int_B f~d\lambda_d=\int_{B_{i_1}}\cdots\int_{B_{i_d}} f(x_1,\ldots,x_n)
	d\lambda(x_{i_d})\ldots d\lambda(x_{i_1})
\]
für jede Permutation $(i_1,\ldots,i_d)$ von $(1,\ldots,n)$.

\textbf{Substitutionsregel}\\
Für $a,b\in\R$ mit $a<b$, $g:[a,b]\to I$ stetig diferenzierbar, $f:I\to\R$ stetig gilt
\[
	\int_a^b f(g(y))\cdot g'(y)dy=\int_{g(a)}^{g(b)}f(x)dx.
\]

\textbf{Partielle Integration}\\
Für $a,b\in\R$ mit $a<b$ und $f,g:[a,b]\to\R$ stetig differenzierbar gilt
\[
	\int_a^b f'(x)\cdot g(x)dx 
	= f(x)\cdot g(x)\Big\vert_a^b-\int_a^b f(x)\cdot g'(x)dx.
\]

\section{Zufallsvariablen}
$X:\Omega\to\R$ ist \textbf{reellwertige Zufallsvariable}, wenn
\[\forall I \text{ Intervall}: \{\omega\in\Omega\mid X(\omega)\in I\}\in\A.\]
$X:\Omega\to\R^d$ ist \textbf{reellwertiger Zufallsvektor}, wenn jede Komponente Zufallsvariable ist.\\
\ \newline
Nützlich hierbei:\\
$\forall x\in\R:\{X\leq x\}\in\A
\Leftrightarrow\forall I \text{ Intervall}:\{X\in I\}\in\A$\\
\ \newline
\textbf{Verteilungsfunktion}: $\F_X:\R\to[0,1],\ x\mapsto\P(\{X\leq x\})$\\
ist monoton wachsend, rechtsseitig stetig,$\lim\limits_{x\to\infty}\P(x)=1$,
$\lim\limits_{x\to -\infty}\P(x)=0$, $\F_X(x)-\F_X(x\mathunderscore)=\P(\{X=x\})$\\
\ \newline
\textbf{Verteilung}: $\P_X: \B{d}\to [0,1],\ B\mapsto\P(\{X\in B\})$\\
ist W'Maß auf $\B{d}$\\
\ \newline\newline

\begin{table}[h]
\centering
%\caption{Vergleich}
\begin{tabular}{P{0.5\linewidth} P{0.5\linewidth}}
\textbf{diskret} & \textbf{absolut stetig} \\
\multicolumn{2}{c}{\textit{Definition}}  \\
$X$ diskret, wenn $\P(\{X\in D\})=1$ für abzählbares $D\subset\R$ (kleinstes solches $D=:D_X$ heißt Träger von $X$.) & \\
Dann existiert Wahrscheinlichkeitsfunktion:
\[f_X:D_X\to [0,1],\ x\mapsto \P(\{X=x\})\] &        
$X$ absolut stetig, wenn $\P_X$ Dichte $f_X$ besitzt\\
\multicolumn{2}{c}{\textit{Verteilungsfunktion}}  \\
$\F_X(x)=\sum\limits_{y\in D_X\cap (-\infty, x]} \P(\{X=y\})$ &
$\F_X(x)=\int\limits_{(-\infty,x]}f_X(y) d\lambda(y)$ \\
\multicolumn{2}{c}{\textit{Verteilung}}  \\
$\P_X(A)=\sum\limits_{x\in D_X\cap A} \P(\{X=x\})$ &
$\P_X(A)=\int\limits_{A} f_X(x) d\lambda(x)$ \\
        &               
\end{tabular}
\end{table}




\section{Verteilungen}

\textbf{Bernoulli-Verteilung} $X\sim\mathbf{B}(1,p)$ mit $p\in [0,1]$
\begin{itemize}
\item $\P(\{X=1\})=p,\ \P(\{X=0\})=1-p$

\item diskret mit $D_X=\begin{cases}
\{0,1\} 	& \text{falls } p\in (0,1)	\\
\{0\}	& \text{falls } p=0			\\
\{1\} 	& \text{falls } p=1
\end{cases}$

\item $\F_X(x)=\begin{cases}
0 	& \text{für } x<0				\\
1-p	& \text{für } 0\leq x <1			\\
1 	& \text{für } x\geq 1
\end{cases}$

\item $\E(X)=p$, $\Var(X)=p-p^2=p(1-p)$

\end{itemize}

\textbf{Binomial-Verteilung} $X\sim\mathbf{B}(n,p)$ mit $p\in [0,1]$
\begin{itemize}
\item $\forall k\in\{0,\ldots,n\}:\P(\{X=k\})=\binom{n}{k}\cdot p^k\cdot (1-p)^{n-k}$

\item diskret mit
$D_X=\begin{cases}
\{0,\ldots,n\} 	& \text{falls } p\in (0,1)	\\
\{0\}			& \text{falls } p=0			\\
\{n\} 			& \text{falls } p=1
\end{cases}$

\item $\E(X)=n\cdot p$, $\Var(X)=n\cdot p\cdot (1-p)$

\item Anwendung: Zählen der Erfolge von $n$ unabhängigen, hintereinander ausgeführten
Experimenten mit Erfolgswahrscheinlichkeit $p_n$
\end{itemize}

\textbf{Hypergeometrische Verteilung} $X\sim\mathbf{H}(N,N_0,n)$
\begin{itemize}
\item $\forall l\in D_X:\P(\{X=l\})
=\frac{\binom{N_0}{l}\cdot\binom{N-N_0}{n-l}}{\binom{N}{n}}$

\item diskret mit 
$D_X=\{\operatorname{max}(0,n-(N-N_0)),\ldots,\operatorname{min}(N_0,n)\}$

\item Anwendung: unter $N$ Objekten finden sich $N_0$ markierte und es werden
$n$ entnommen
\end{itemize}

\textbf{Poisson-Verteilung} $X\sim\mathbf{P}(\lambda)$ für $\lambda>0$
\begin{itemize}
\item $\forall k\in\N_0:\P(\{X=k\})=\exp(-\lambda)\cdot\frac{\lambda^k}{k!}$

\item diskret mit $D_X=\N_0$

\item $X\sim\mathbf{P}(\lambda_2), Y\sim\mathbf{P}(\lambda_2)
\Rightarrow X+Y\sim\mathbf{P}(\lambda_2 + \lambda_2)$ \U

\item $\E(X)=\lambda=\Var(X)$, $\E(X^2)=\lambda^2+\lambda$

\item Anwendung: Approximation von $\mathbf{B}(n,p)$ durch $\mathbf{P}(\lambda)$
mit $\lambda=n\cdot p$ für 'große' $n$ und 'kleine' $p$, also Eintreffen eines
seltenen Ereignis bei großer Anzahl an Wiederholungen
\end{itemize}

\newpage
\textbf{Geometrische Verteilung} $X\sim\mathbf{G}(p)$ mit $p\in\textbf{(}0,1]$
\begin{itemize}
\item $\forall k\in\N:\P(\{X=k\})=p\cdot(1-p)^{k-1}$

\item diskret mit $D_X=\N$

\item $\E(X)=\frac{1}{p}$, $\Var(X)=\frac{1-p}{p^2}$, $\E(X^2)=\frac{2-p}{p^2}$

\item Gedächtnislos:\\ $\forall k_1,k_2\in\N \text{ mit } k_1<k_2:
\P(\{X>k_2\}\mid \{ X>k_1\}) = \P(\{X>k_2-k_1\})$

\item Anwendung: diskretes Warten bis zum ersten Eintritt eines Ereignisses
\end{itemize}

\textbf{Gleichverteilung} $X\sim\mathbf{U}([a,b])$ für $-\infty<a<b<\infty$
\begin{itemize}
\item absolut stetig mit 
$f_X(x)=\begin{cases}
\frac{1}{b-a} 	& \text{falls } x\in [a,b]	\\
0				& \text{sonst }		
\end{cases}$

\item
$\F_X(x)=\begin{cases}
0 				& \text{falls } x<a			\\
\frac{x-a}{b-a}	& \text{falls } x\in [a,b]	\\
1				& \text{sonst}
\end{cases}$

\item $\E(X)=\frac{a+b}{2}$, $\Var(X)=\frac{(b-a)^2}{12}$,
$\E(X^2)=\frac{b^3-a^3}{3(b-a)}$
\end{itemize}

\textbf{Einpunktverteilung} $X\sim\mathbf{U}(\{c\})$ für ein $c\in\R$
\begin{itemize}
\item $P(\{X=c\})=c$

\item diskret mit $D_X=\{c\}$

\item $\F_X(x)=\begin{cases}
0 	&\text{falls } x<c			\\
1	& \text{falls } x\geq c
\end{cases}$

\item $\E(X)=c$, $\Var(X)=0$
\end{itemize}

\textbf{Exponentialverteilung} $X\sim\mathbf{Exp}(\lambda)$ für $\lambda>0$
\begin{itemize}
\item absolut stetig mit
$f_X:\R\to [0,\infty],\ x\mapsto\begin{cases}
\lambda\cdot\exp(-\lambda x) 	& \text{falls } x\geq 0	\\
0							& \text{sonst }		
\end{cases}$

\item $\F_X(x)=1-\exp(-\lambda x)$

\item $\E(X)=\frac{1}{\lambda}$, $\Var(X)=\frac{1}{\lambda^2}$,
$\E(X^2)=\frac{2}{\lambda^2}$

\item Gedächtnislos: $\forall s,t>0:\P(\{X>t+s\}\mid\{X>t\})=\P(\{X>s\})$

\item Anwendung: Warten bis zum ersten Eintritt eines Ereignisses (zB Lebensdauer,
radioaktiver Zerfall,\ldots)
\end{itemize}

\textbf{Standard-Normalverteilung} $X\sim\mathbf{N}(0,1)$
\begin{itemize}
\item absolut stetig mit $f_X:\R\to[0,\infty],\
x\mapsto\frac{1}{\sqrt{2\pi}}\cdot\exp(-\frac{x^2}{2})$

\item $\Phi(x):=\F_X(x)=
\frac{1}{\sqrt{2\pi}}\int_{-\infty}^{x}\exp(-\frac{y^2}{2})d\lambda(y)$
(keine explizite Formel)

\item $\forall z\in\R:\Phi(z)+\Phi(-z)=1$ \U

\item $\E(X)=0$, $\Var(X)=1$

\end{itemize}

\textbf{Normalverteilung} $X\sim\mathbf{N}(\mu,\sigma^2)$ für 
$\mu\in\R,\ \sigma>0$
\begin{itemize}
\item absstetig mit $f_X:\R\to[0,\infty],\ 
x\mapsto\frac{1}{\sqrt{2\pi\sigma^2}}\cdot\exp(-\frac{(x-\mu)^2)}{2\sigma^2})$

\item $X\sim\mathbf{N}(\mu,\sigma^2)
\Rightarrow a\cdot X+b\sim\mathbf{N}(a\cdot\mu+b,a^2\cdot\sigma^2)$

\item Also: $\frac{X-\mu}{\sigma}\sim\mathbf{N}(0,1)$

\item $\E(X)=\mu$, $\Var(X)=\sigma^2$
\end{itemize}

\textbf{Gleichverteilung} (mehrdimensional) $X\sim\mathbf{U}(S)$ für $S\subset\R^n$
\underline{endlich}
\begin{itemize}
\item $\forall x\in S: \P(\{X=x\})=\frac{1}{|S|}$

\item diskret mit $D_X=S$
\end{itemize}

\textbf{Gleichverteilung} (mehrdimensional) $X\sim\mathbf{U}(G)$ für $G\in\B{d}$
\begin{itemize}
\item $\P_X(A)=\frac{\lambda_d(A\cap G)}{\lambda_d(G)}=
\int_A \frac{1}{\lambda_d(G)\cdot 1_G(x)}d\lambda(x)$

\item absolut stetig mit $f_X=\frac{1}{\lambda_d(G)}\cdot \1_G$
\end{itemize}

\textbf{Standard-Normalverteilung} (mehrdimensional)
\begin{itemize}
\item absolut stetig mit 
$f_X:\R^d\to [0,\infty),\ 
x\mapsto(2\pi)^{-d/2}\cdot\exp(-\frac{1}{2}\sum_{i=1}^{d}x_i^2)$

\item $X=(X_1,\ldots,X_d)$ ist standard-normalverteilt 
$\Leftrightarrow$ $X_1,\ldots,X_d\ \operatorname{i.i.d}\sim\mathbf{N}(0,1)$
\end{itemize}

%\section{Erwartungswert \& Varianz}

\textbf{Integriebare Zufallsvariablen}
\begin{itemize}
\item Diskretes $X$ mit Träger $D_X$ \textit{integrierbar}
$:\Leftrightarrow \sum_{x\in D_X} |x|\cdot \P(\{X=x\})<\infty$\\
Diskrete Zufallsvariablen mit \underline{endlichem} Träger sind immmer integrierbar!

\item  Absolut stetige Zufallsvariable $X$ mit Dichte $f_X$ \textit{integrierbar}
\mbox{$:\Leftrightarrow \int_\R |x|\cdot f_X(x)d\lambda(x)<\infty$}

\item $\L_1:=\{X \text{ integrierbar}\}$ ist Vektorraum

\item \textit{Erwartungswert $\E(X)$} für diskretes $X\in\L_1$:
$\E(X):= \sum_{x\in D_X} x\cdot\P(\{X=x\})\in\R$

\item \textit{Erwartungswert $\E(X)$} für absolut stetiges $X\in\L_1$:
$\E(X):= \int_\R x\cdot f_X~d\lambda(x)\in\R$

\item Der Erwartungswert ist linear und monoton.

\item \textbf{Transformationssatz}: Für $X$ $d$-dimensionaler Zufallsvektor diskret/
absolut stetig und \mbox{$h:\R^d\to\R$ Borel-messbar} gilt
\[
	h(X)\in\L_1 \Leftrightarrow
	\begin{cases}
	\sum_{x\in D_X} |h(x)|\cdot\P(\{X=x\})<\infty			& \text{X diskret} \\
	\int_{\R^d} |h(x)|\cdot f_X(x)~d\lambda_d(x)<\infty	& \text{X absolut stetig}
	\end{cases}.
\]
Gegebenenfalls
\[
	\E(h(X))=
	\begin{cases}
	\sum_{x\in D_X} h(x)\cdot\P(\{X=x\})<\infty			& \text{X diskret} \\
	\int_{\R^d} h(x)\cdot f_X(x)~d\lambda_d(x)<\infty		& \text{X absolut stetig}
	\end{cases}.
\]

\item $X,Y\in\L_1$ \hypertarget{unabhaengig}{unabhängig}
$\Rightarrow X\cdot Y\in\L_1$ mit $\E(X\cdot Y) = \E(X)\cdot \E(Y)$

\end{itemize}
\hspace{3em}

\textbf{Quadratisch integrierbare Zufallsvariablen}
\begin{itemize}
\item $X$ \textit{quadratisch integrierbar} $:\Leftrightarrow X^2\in\L_1$

\item $\L_2:=\{X \text{ quadratisch integrierbar}\}$ ist Untervektorraum von $\L_1$

\item Für $X$ diskret mit Träger $D_X$ bzw. absolut stetig mit Dichte $f_X$ gilt:
\[
	X\in\L_2\Leftrightarrow
	\begin{cases}
	\sum_{x\in D_X} x^2\cdot\P(\{X=x\})<\infty			& \text{X diskret} \\
	\int_{\R^d} x^2\cdot f_X(x)~d\lambda_d(x)<\infty		& \text{X absolut stetig}
	\end{cases}.
\]
Gegebenenfalls ist $\E(X^2)$ durch obige(s) Summe/Integral gegeben.

\item Für $X\in\L_2$ ist $\Var(X):=\E((X-\E(X))^2)$ die \textit{Varianz} von $X$.

\item $\Var(X)=\E(X^2)-(\E(X))^2$, \quad $\Var(\alpha\cdot X+\beta)=\alpha^2\Var(X)$

\item \textbf{Tschebyschev-Ungleichung}. Für $X\in\L_2, \varepsilon>0$ gilt:
\[
	\P(|X-\E(X)|\geq \varepsilon)\leq \frac{1}{\varepsilon^2}\cdot\Var(X)
\]
sowie
\[
	\Var(X)=0 \Leftrightarrow \P(\{X=\E(x)\})=1.
\]

\item \textbf{Formel von Bienaymé}. Für $X_1,\ldots,X_n\in\L_2$ unabhängig gilt:
\[
	\Var(\sum_{i=1}^n X_i) = \sum_{i=1}^n\Var(X_i).
\]

\item $X,Y\in\L_2\Rightarrow X\cdot Y\in\L_1$

\item $\overline{\L_2}$ ist Hilbertraum mit $\langle X,Y\rangle :=\E(X\cdot Y)$
\end{itemize}
\hspace{3em}

\textbf{Kovarianz}
\begin{itemize}
\item Für $X,Y\in\L_2$ ist die \textit{Kovarianz} definiert durch
\mbox{$\Cov(X,Y):=\E\big((X-\E(X)\cdot (Y-\E(Y))\big)$}

\item Für $X,Y\in\L_2$ gilt $\Cov(X,Y)=\E(X\cdot Y)-\E(X)\cdot\E(Y)$

\item $X,Y$ heißen \textit{unkorreliert} wenn $\Cov(X,Y)=0$.
\mbox{Es gilt: $X,Y$ unabhängig $\Rightarrow X,Y$ unkorreliert}

\item Für $X,Y$ mit $\Var(X),\Var(Y)>0$ ist
\[
	\rho(X,Y):=\frac{\Cov(X,Y)}{\sqrt{\Var(X)\cdot\Var(Y)}} = 
	\cos(\sphericalangle (X,Y))
\]
der \textit{Korrelationskoeffizient} von $X$ und $Y$.

\item \textbf{Cauchy-Schwarz}: $X,Y\in\L_2
\Rightarrow |\E(X\cdot Y)|\leq \sqrt{\E(X^2)\cdot\E(Y^2)}$
\end{itemize}


%\include{sections/grenzwerte}
\section{Sonstiges}


\textbf{Exponentialfunktion}:
\begin{itemize}
\item $\exp:\mathbb{C}\to\mathbb{C},\ x\mapsto\sum_{k=0}^\infty \frac{z^k}{k!}$

\item $\exp(a\cdot b)=\exp(a)+\exp(b)$

\item $\exp$ stetig, $\exp>0$

\item $\exp'=\exp$
\end{itemize}

\textbf{Natürlicher Logarithmus}
\begin{itemize}
\item $\ln: (0,\infty)\to\R:\ x\mapsto\exp^{-1}(x)$

\item $\ln(a\cdot b)=\ln(a)+\ln(b)$

\item $\ln(x)=\frac{1}{x}$
\end{itemize}

\textbf{Trigonometrische Funktionen}
\begin{itemize}
\item $\cos:\R\to\R:\ x\mapsto\sum_{k=0}^\infty (-1)^k\cdot\frac{x^{2k}}{(2k)!}$

\item $\sin:\R\to\R:\ x\mapsto\sum_{k=0}^\infty (-1)^k\cdot\frac{x^{2k+1}}{(2k+1)!}$

\item $\cos' = -\sin$, $\sin' = \cos$
\end{itemize}

\textbf{Reihen}
\begin{itemize}
\item \textbf{Harmonische Reihe}: $\sum_{n=1}^\infty \frac{1}{n}=\infty$

\item \textbf{Geometrische Reihe}: $\sum_{n=0}^\infty a\cdot x^n = \frac{a}{1-x}$
für $a,x\in\R$ mit $|x|<1$, divergent für alle anderen $x\in\R$

\item \textbf{Alternierende harmonische Reihe}:
$\sum_{n=1}^\infty (-1)^n\cdot\frac{1}{n}=-\ln(2)$
\end{itemize}

\end{document}
