\documentclass[11pt,a4paper,ngerman]{article}

% IMPORTS
\usepackage[ngerman]{babel}
\usepackage[utf8x]{inputenc}
\usepackage[colorinlistoftodos,prependcaption,textsize=tiny]{todonotes} % To-Dos
\usepackage{parskip} 	% Absätze ohne Einrückungen
\usepackage{mathtools} 	% Für Pfeil mit Erklärung
\usepackage{amsfonts} 	% Buchstaben mit Doppelstrich
\usepackage{amsmath}
\usepackage{amssymb}
\usepackage{amsthm}		
\usepackage{bbm}			% Indikator Funktion
\usepackage{verbatim}	% comment-env
\usepackage{wasysym}		% \lightning
\usepackage{graphicx}	% Einbinden von Grafiken
\usepackage{url}
\usepackage{hyperref}

% NEWCOMMANDS
\newcommand{\A}{\mathcal{A}}
\newcommand{\compl}[1]{{#1}^{\text{\tiny C}}}
\renewcommand{\P}{\operatorname{P}}

% META
\date{07.02.2018}
\author{Maximilian Reif}
\title{Notizen zur Einführung in die Stochastik}

%--------------------------------------------------------------------------------------------------------------

\begin{document}

\begin{titlepage}
    \ \newline\newline\newline\newline\newline
	
	\begin{center}

		\huge Notizen zu\\
		\Huge\textbf{Einführung in die Stochastik}\\
		\huge bei Prof. Kaiser im WS 17/17\\
		\normalsize

		\vspace{1cm}
		\begin{tabular}[b]{l|l}
			\textbf{author} 		& Maximilian Reif
			\texttt{\href{mailto:reifmaxi@fim.uni-passau.de}
			{<reifmaxi@fim.uni-passau.de>}}\\
			\textbf{last change}	& \today \\
			\textbf{version} 	& 0.0.1\\
			\textbf{github} 		& \url{https://github.com/lordreif/stochastik}
		\end{tabular}
		\vspace{1cm}
		
	\end{center}
	
	\begin{figure}[b]
	\centering
	\includegraphics[width=0.6\textwidth]{img/fim.png}
	\end{figure}
	
\end{titlepage}

%\include{sections/00-annotation}

\newpage
%\tableofcontents\thispagestyle{empty}
\newpage

\setcounter{page}{1}
\section{Basics}

\begin{itemize}

\item \mbox{\textbf{Ergebnis:} Element aus $\Omega$, 
\textbf{Ereignis: } Element aus $\A$ (Teilmenge von $\Omega$)}

\item \textbf{$\sigma$-Algebra $\A$}
	\begin{enumerate}
	\item $\Omega\in\A$
	\item $A\in\A \Rightarrow \compl{A}\in\A$
	\item $A_1, A_2,\ldots\in\A \Rightarrow \bigcup_{i=1}^{\infty} A_i \in\A$
	\end{enumerate}
	
\item \textbf{Wahrscheinlichkeitsmaß:} $\P:\A\rightarrow [0,1]$
	\begin{enumerate}
	\item $\P(\Omega)=1$
	\item $A_1,A_2;\ldots\in\A$ \underline{p.d.} 
	$\Rightarrow \P(\bigcup_{i=1}^{\infty} A_i)
	= \sum_{i=1}^{\infty}\P(A_i)$ \quad\textbf{($\sigma$-Additivität)}
	\end{enumerate}

\item $A,B\in\A:$
	\begin{enumerate}
	\item $A\subset B \Rightarrow \P(B)=\P(A)+\P(B\backslash A)$
	\item $A\subset B \Rightarrow \P(A)\leq\P(B)$ \qquad\textbf{(Monotonie des W'Maßes)}
	\item $\P(A^{\text{C}})=1-\P(A)$
	\item $\P(A\cup B) = \P(A)+\P(B)-\P(A\cap B)$
	\end{enumerate}
	
\item $A_1,A_2,\ldots\in\A$
	\begin{enumerate}
	\item $\P(\bigcup_{i=1}^{\infty} A_i)\leq\sum_{i=1}^{\infty} \P(A_i)$ 
	\qquad\textbf{($\sigma$-Subadditivität)}
	\item $A_1\subset A_2\subset\ldots \Rightarrow 
	\P(\bigcup_{i=1}^{\infty} A_i)=\lim\limits_{i\to\infty} \P(A_i)$
	\mbox{(\textbf{$\sigma$-Stetigkeit von oben})}
	\item $A_1 \supset A_2\supset\ldots \Rightarrow
	\P(\bigcap_{i=1}^{\infty} A_i)=\lim\limits_{i\to\infty} \P(A_i)$
	\mbox{(textbf{$\sigma$-Stetigkeit von unten})}
	\end{enumerate}

\item $\P(B)>0$, dann $\P(\ \cdot\mid B) := \frac{\P(\ \cdot\ \cap B)}{\P(B)}$
W'Maß auf $\A$ mit $\P(B\mid B)=1$

\item \textbf{Formel der totalen Wahrscheinlichkeit}
\[
	B_1, B_2,\ldots \in\A \text{ \underline{p.d.} mit } \P(B_i)>0, 
	\bigcup_{i=1}^{n} B_i = \Omega
\]
\[
	\Rightarrow \P(A)=\sum_{i=1}^{n} \P(A\mid B_i)\cdot\P(B_i)
	\text{ für alle } A\in\A
\]

\item \textbf{Formel von Bayes}
\[
	B_1, B_2,\ldots \in\A \text{ \underline{p.d.} mit } \P(B_i)>0, 
	\bigcup_{i=1}^{n} B_i = \Omega
\]
\[
	\Rightarrow\P(B_i\mid A)=
	\frac{\P(A\mid B_i)\cdot\P(B_i)}{\sum\limits_{j=1}^{n} \P(A\mid B_j)\cdot\P(B_j)}\ 
	\text{für alle } A\in\A \text{ mit } \P(A)>0
\]

\item $\P(B\mid A)=\frac{\P(A\mid B)\cdot\P(B)}{\P(A)}$ für $A,B\in\A$ mit
$\P(A),\P(B)>0$

\item $\P(A\cap B)=\P(A\mid B)\cdot\P(B)$ für $B\in\A$ mit $\P(B)>0$

\item $A_1,\ldots ,A_n\in\A$ mit $\P(\bigcap_{i=1}^{n-1} A_i)>0$, dann $\P(\bigcap_{i=1}^{n} A_i)=$\\ 
$P(A_1)\cdot\P(A_2\mid A_1)\cdots
\P(A_1\cap\ldots\cap A_n)\cdot\P(A_n\mid A_1\cap\ldots\cap A_{n-1})$

\item $A,B$ unabhängig $:\Leftrightarrow \P(A\cap B)=\P(A)\cdot\P(B)
\Leftrightarrow P(A\mid B) = P(A)$ falls $\P(B)>0$

\item $A_1,\ldots ,A_n$ unabhängig 
$\Leftrightarrow\P(\bigcap\limits_{i=1}^{n})=\prod_{i=1}^{n} P(A_i)
\Rightarrow$ paarweise Unabhängigkeit

\item $\binom{n}{k}:= \frac{n!}{k!(n-k)!}$, 
$\binom{n}{k-1}+\binom{n}{k}=\binom{n+1}{k}$,
$\binom{n}{k}=\binom{n}{n-k}$

\item $ |N_1 \times N_2 \times\dots\times N_k| = |N_1| \cdot |N_2| \cdots |N_k|$,
speziell $|N^k| = {|N|}^k$

\item \mbox{$n=|N|,k\leq n$, dann $|\{(\omega_1,\ldots,\omega_k)\in N^k 
\mid\omega_i\neq\omega_j\text { für }i\neq j\}|=\frac{n!}{(n-k)!}$}

\item $n=|N|,k\leq n$, dann $|\{K\subset N\mid |K|=k\}| = \binom{n}{k} =
|\{(\omega_1,\ldots,\omega_k)\in N^k \mid 1\leq\omega_1<\ldots<\omega_k\leq n\}|$

\item $k,n\in\mathbb{N}$, dann 
$|\{(\omega_1,\ldots,\omega_k)\in N^k\mid 1\leq\omega_1\leq\ldots\leq\omega_k\leq n\}|
=\binom{n+k-1}{k}$

\item Insgesamt:

\begin{table}[h]
\centering
\begin{tabular}{l|l|l|}
\cline{2-3}
                                       & \begin{tabular}[c]{@{}l@{}}Berücksichtigung\\ der Reihenfolge\end{tabular} & \begin{tabular}[c]{@{}l@{}}keine Berücksichtigung\\ der Reihenfolge\end{tabular}                                                                   \\ \hline
\multicolumn{1}{|l|}{mit Zurücklegen}  & \begin{tabular}[c]{@{}l@{}}$\Omega=N^k$,\\ $|\Omega = n^k|$\end{tabular}   & \begin{tabular}[c]{@{}l@{}}$\Omega = \{\omega\in N^k\mid1\leq\omega_1\leq\ldots\leq\omega_n\leq n\}$,\\ $|\Omega| = \binom{n+k-1}{k}$\end{tabular} \\ \hline
\multicolumn{1}{|l|}{ohne Zurücklegen} & \begin{tabular}[c]{@{}l@{}}$\Omega=\{\omega\in N^k \mid \omega_i\neq\omega_j \text{ für } i\neq j\} $\\ $|\Omega|=\frac{n!}{(n-k)!}$\end{tabular}  & \begin{tabular}[c]{@{}l@{}}$\Omega=\{\omega\in N^k\mid 1\leq\omega_1<\ldots<\omega_k\leq n\}$,\\ $|\Omega|=\binom{n}{k}$\end{tabular}              \\ \hline
\end{tabular}
\end{table}

\end{itemize}
%\section{Zufallsvariablen}
$X:\Omega\to\R$ ist \textbf{reellwertige Zufallsvariable}, wenn
\[\forall I \text{ Intervall}: \{\omega\in\Omega\mid X(\omega)\in I\}\in\A.\]
$X:\Omega\to\R^d$ ist \textbf{reellwertiger Zufallsvektor}, wenn jede Komponente Zufallsvariable ist.\\
\ \newline
Nützlich hierbei:\\
$\forall x\in\R:\{X\leq x\}\in\A
\Leftrightarrow\forall I \text{ Intervall}:\{X\in I\}\in\A$\\
\ \newline
\textbf{Verteilungsfunktion}: $\F_X:\R\to[0,1],\ x\mapsto\P(\{X\leq x\})$\\
ist monoton wachsend, rechtsseitig stetig,$\lim\limits_{x\to\infty}\P(x)=1$,
$\lim\limits_{x\to -\infty}\P(x)=0$, $\F_X(x)-\F_X(x\mathunderscore)=\P(\{X=x\})$\\
\ \newline
\textbf{Verteilung}: $\P_X: \B{d}\to [0,1],\ B\mapsto\P(\{X\in B\})$\\
ist W'Maß auf $\B{d}$\\
\ \newline\newline

\begin{table}[h]
\centering
%\caption{Vergleich}
\begin{tabular}{p{0.5\linewidth} p{0.5\linewidth}}
\textbf{diskret} & \textbf{absolut stetig} \\
\multicolumn{2}{c}{\textit{Definition}}  \\
$X$ diskret, wenn $\P(\{X\in D\})=1$ für abzählbares $D\subset\R$ (kleinstes solches $D=:D_X$ heißt Träger von $X$.) & \\
Dann existiert Wahrscheinlichkeitsfunktion:
\[f_X:D_X\to [0,1],\ x\mapsto \P(\{X=x\})\] &        
$X$ absolut stetig, wenn $\P_X$ Dichte $f_X$ besitzt\\
\multicolumn{2}{c}{\textit{Verteilungsfunktion}}  \\
$\F_X(x)=\sum\limits_{y\in D_X\cap (-\infty, x]} \P(\{X=y\})$ &
$\F_X(x)=\int\limits_{(-\infty,x]}f_X(y) d\lambda(y)$ \\
\multicolumn{2}{c}{\textit{Verteilung}}  \\
$\P_X(A)=\sum\limits_{x\in D_X\cap A} \P(\{X=x\})$ &
$\P_X(A)=\int\limits_{A} f_X(x) d\lambda(x)$ \\
        &               
\end{tabular}
\end{table}




%\include{sections/03-erwart_var}
%\include{sections/04-grenzwert}

\end{document}