\documentclass[11pt,a4paper,ngerman]{article}

% IMPORTS
\usepackage[ngerman]{babel}
\usepackage[utf8x]{inputenc}
\usepackage[colorinlistoftodos,prependcaption,textsize=tiny]{todonotes} % To-Dos
\usepackage{parskip} 	% Absätze ohne Einrückungen
\usepackage{mathtools} 	% Für Pfeil mit Erklärung
\usepackage{amsfonts} 	% Buchstaben mit Doppelstrich
\usepackage{amsmath}
\usepackage{amssymb}
\usepackage{amsthm}		
\usepackage{bbm}			% Indikator Funktion
\usepackage{verbatim}	% comment-env
\usepackage{wasysym}		% \lightning
\usepackage{graphicx}	% Einbinden von Grafiken
\usepackage[left=3cm,right=3cm,top=3cm,bottom=3.5cm,includeheadfoot]{geometry}
\usepackage{url}
\usepackage{hyperref}

% NEWCOMMANDS
\newcommand{\A}{\mathcal{A}}
\newcommand{\B}[1]{\mathcal{B}_{#1}}
\renewcommand{\P}{\operatorname{P}}
\newcommand{\F}{\operatorname{F}}
\newcommand{\Var}{\operatorname{Var}}
\newcommand{\E}{\operatorname{E}}
\newcommand{\R}{\mathbb{R}}
\newcommand{\N}{\mathbb{N}}
\newcommand{\1}{\mathbbm{1}}
\newcommand{\compl}[1]{{#1}^{\text{\tiny C}}}
\newcommand{\U}{\texttt{(Übung)}}

% META
\date{07.02.2018}
\author{Maximilian Reif}
\title{Notizen zur Einführung in die Stochastik}

%--------------------------------------------------------------------------------------------------------------

\begin{document}

\begin{titlepage}
    \ \newline\newline\newline\newline\newline
	
	\begin{center}

		\huge Notizen zu\\
		\Huge\textbf{Einführung in die Stochastik}\\
		\huge bei Prof. Kaiser im WS 17/18\\
		\normalsize

		\vspace{1cm}
		\begin{tabular}[b]{l|l}
			\textbf{author} 		& Maximilian Reif
			\texttt{\href{mailto:reifmaxi@fim.uni-passau.de}
			{<reifmaxi@fim.uni-passau.de>}}\\
			\textbf{last change}	& 8. Feb 2018 \\
			\textbf{version} 	& 0.0.3\\
			\textbf{github} 		& \url{https://github.com/lordreif/stochastik}
		\end{tabular}
		\vspace{1cm}
		
	\end{center}
	
	\begin{figure}[b]
	\centering
	\includegraphics[width=0.6\textwidth]{img/fim.png}
	\end{figure}
	
\end{titlepage}

%\section*{Disclaimer}
\thispagestyle{empty}

Dies ist eine Kurzzusammenfassung des Stoffes der Vorlesung
\textit{Einführung in die Stochastik} im Wintersemester 17/18 an der Universität Passau.

Das Dokument erhebt nicht den Anspruch, in sich geschlossen zu sein, und verzichtet
gänzlich auf Beweise.\\
Da die Verständlichkeit über Formales gestellt wurde, besteht weder auf mathematische
Richtigkeit noch auf Vollständigkeit Anspruch.


Falls Sie Fehler in Layout und Sprache finden, oder sich falsche Behauptungen
eingeschlichen haben, teilen Sie dies bitte dem Autor mit.


\newpage
%\tableofcontents\thispagestyle{empty}
\newpage

\setcounter{page}{1}
\section{Basics}

\begin{itemize}

\item \mbox{\textbf{Ergebnis:} Element aus $\Omega$, 
\textbf{Ereignis: } Element aus $\A$ (Teilmenge von $\Omega$)}

\item \textbf{$\sigma$-Algebra $\A$}
	\begin{enumerate}
	\item $\Omega\in\A$
	\item $A\in\A \Rightarrow \compl{A}\in\A$
	\item $A_1, A_2,\ldots\in\A \Rightarrow \bigcup_{i=1}^{\infty} A_i \in\A$
	\end{enumerate}
	
\item \textbf{Wahrscheinlichkeitsmaß:} $\P:\A\rightarrow [0,1]$
	\begin{enumerate}
	\item $\P(\Omega)=1$
	\item $A_1,A_2;\ldots\in\A$ \underline{p.d.} 
	$\Rightarrow \P(\bigcup_{i=1}^{\infty} A_i)
	= \sum_{i=1}^{\infty}\P(A_i)$ \quad\textbf{($\sigma$-Additivität)}
	\end{enumerate}

\item $A,B\in\A:$
	\begin{enumerate}
	\item $A\subset B \Rightarrow \P(B)=\P(A)+\P(B\backslash A)$
	\item $A\subset B \Rightarrow \P(A)\leq\P(B)$ \qquad\textbf{(Monotonie des W'Maßes)}
	\item $\P(A^{\text{C}})=1-\P(A)$
	\item $\P(A\cup B) = \P(A)+\P(B)-\P(A\cap B)$
	\end{enumerate}
	
\item $A_1,A_2,\ldots\in\A$
	\begin{enumerate}
	\item $\P(\bigcup_{i=1}^{\infty} A_i)\leq\sum_{i=1}^{\infty} \P(A_i)$ 
	\qquad\textbf{($\sigma$-Subadditivität)}
	\item $A_1\subset A_2\subset\ldots \Rightarrow 
	\P(\bigcup_{i=1}^{\infty} A_i)=\lim\limits_{i\to\infty} \P(A_i)$
	\mbox{(\textbf{$\sigma$-Stetigkeit von oben})}
	\item $A_1 \supset A_2\supset\ldots \Rightarrow
	\P(\bigcap_{i=1}^{\infty} A_i)=\lim\limits_{i\to\infty} \P(A_i)$
	\mbox{(textbf{$\sigma$-Stetigkeit von unten})}
	\end{enumerate}

\item $\P(B)>0$, dann $\P(\ \cdot\mid B) := \frac{\P(\ \cdot\ \cap B)}{\P(B)}$
W'Maß auf $\A$ mit $\P(B\mid B)=1$

\item \textbf{Formel der totalen Wahrscheinlichkeit}
\[
	B_1, B_2,\ldots \in\A \text{ \underline{p.d.} mit } \P(B_i)>0, 
	\bigcup_{i=1}^{n} B_i = \Omega
\]
\[
	\Rightarrow \P(A)=\sum_{i=1}^{n} \P(A\mid B_i)\cdot\P(B_i)
	\text{ für alle } A\in\A
\]

\item \textbf{Formel von Bayes}
\[
	B_1, B_2,\ldots \in\A \text{ \underline{p.d.} mit } \P(B_i)>0, 
	\bigcup_{i=1}^{n} B_i = \Omega
\]
\[
	\Rightarrow\P(B_i\mid A)=
	\frac{\P(A\mid B_i)\cdot\P(B_i)}{\sum\limits_{j=1}^{n} \P(A\mid B_j)\cdot\P(B_j)}\ 
	\text{für alle } A\in\A \text{ mit } \P(A)>0
\]

\item $\P(B\mid A)=\frac{\P(A\mid B)\cdot\P(B)}{\P(A)}$ für $A,B\in\A$ mit
$\P(A),\P(B)>0$

\item $\P(A\cap B)=\P(A\mid B)\cdot\P(B)$ für $B\in\A$ mit $\P(B)>0$

\item $A_1,\ldots ,A_n\in\A$ mit $\P(\bigcap_{i=1}^{n-1} A_i)>0$, dann $\P(\bigcap_{i=1}^{n} A_i)=$\\ 
$P(A_1)\cdot\P(A_2\mid A_1)\cdots
\P(A_1\cap\ldots\cap A_n)\cdot\P(A_n\mid A_1\cap\ldots\cap A_{n-1})$

\item $A,B$ unabhängig $:\Leftrightarrow \P(A\cap B)=\P(A)\cdot\P(B)
\Leftrightarrow P(A\mid B) = P(A)$ falls $\P(B)>0$

\item $A_1,\ldots ,A_n$ unabhängig 
$\Leftrightarrow\P(\bigcap\limits_{i=1}^{n})=\prod_{i=1}^{n} P(A_i)
\Rightarrow$ paarweise Unabhängigkeit

\item $\binom{n}{k}:= \frac{n!}{k!(n-k)!}$, 
$\binom{n}{k-1}+\binom{n}{k}=\binom{n+1}{k}$,
$\binom{n}{k}=\binom{n}{n-k}$

\item $ |N_1 \times N_2 \times\dots\times N_k| = |N_1| \cdot |N_2| \cdots |N_k|$,
speziell $|N^k| = {|N|}^k$

\item \mbox{$n=|N|,k\leq n$, dann $|\{(\omega_1,\ldots,\omega_k)\in N^k 
\mid\omega_i\neq\omega_j\text { für }i\neq j\}|=\frac{n!}{(n-k)!}$}

\item $n=|N|,k\leq n$, dann $|\{K\subset N\mid |K|=k\}| = \binom{n}{k} =
|\{(\omega_1,\ldots,\omega_k)\in N^k \mid 1\leq\omega_1<\ldots<\omega_k\leq n\}|$

\item $k,n\in\mathbb{N}$, dann 
$|\{(\omega_1,\ldots,\omega_k)\in N^k\mid 1\leq\omega_1\leq\ldots\leq\omega_k\leq n\}|
=\binom{n+k-1}{k}$

\item Insgesamt:

\begin{table}[h]
\centering
\begin{tabular}{l|l|l|}
\cline{2-3}
                                       & \begin{tabular}[c]{@{}l@{}}Berücksichtigung\\ der Reihenfolge\end{tabular} & \begin{tabular}[c]{@{}l@{}}keine Berücksichtigung\\ der Reihenfolge\end{tabular}                                                                   \\ \hline
\multicolumn{1}{|l|}{mit Zurücklegen}  & \begin{tabular}[c]{@{}l@{}}$\Omega=N^k$,\\ $|\Omega = n^k|$\end{tabular}   & \begin{tabular}[c]{@{}l@{}}$\Omega = \{\omega\in N^k\mid1\leq\omega_1\leq\ldots\leq\omega_n\leq n\}$,\\ $|\Omega| = \binom{n+k-1}{k}$\end{tabular} \\ \hline
\multicolumn{1}{|l|}{ohne Zurücklegen} & \begin{tabular}[c]{@{}l@{}}$\Omega=\{\omega\in N^k \mid \omega_i\neq\omega_j \text{ für } i\neq j\} $\\ $|\Omega|=\frac{n!}{(n-k)!}$\end{tabular}  & \begin{tabular}[c]{@{}l@{}}$\Omega=\{\omega\in N^k\mid 1\leq\omega_1<\ldots<\omega_k\leq n\}$,\\ $|\Omega|=\binom{n}{k}$\end{tabular}              \\ \hline
\end{tabular}
\end{table}

\end{itemize}
%\include{sections(xx-integral}
\section{Zufallsvariablen}
$X:\Omega\to\R$ ist \textbf{reellwertige Zufallsvariable}, wenn
\[\forall I \text{ Intervall}: \{\omega\in\Omega\mid X(\omega)\in I\}\in\A.\]
$X:\Omega\to\R^d$ ist \textbf{reellwertiger Zufallsvektor}, wenn jede Komponente Zufallsvariable ist.\\
\ \newline
Nützlich hierbei:\\
$\forall x\in\R:\{X\leq x\}\in\A
\Leftrightarrow\forall I \text{ Intervall}:\{X\in I\}\in\A$\\
\ \newline
\textbf{Verteilungsfunktion}: $\F_X:\R\to[0,1],\ x\mapsto\P(\{X\leq x\})$\\
ist monoton wachsend, rechtsseitig stetig,$\lim\limits_{x\to\infty}\P(x)=1$,
$\lim\limits_{x\to -\infty}\P(x)=0$, $\F_X(x)-\F_X(x\mathunderscore)=\P(\{X=x\})$\\
\ \newline
\textbf{Verteilung}: $\P_X: \B{d}\to [0,1],\ B\mapsto\P(\{X\in B\})$\\
ist W'Maß auf $\B{d}$\\
\ \newline\newline

\begin{table}[h]
\centering
%\caption{Vergleich}
\begin{tabular}{p{0.5\linewidth} p{0.5\linewidth}}
\textbf{diskret} & \textbf{absolut stetig} \\
\multicolumn{2}{c}{\textit{Definition}}  \\
$X$ diskret, wenn $\P(\{X\in D\})=1$ für abzählbares $D\subset\R$ (kleinstes solches $D=:D_X$ heißt Träger von $X$.) & \\
Dann existiert Wahrscheinlichkeitsfunktion:
\[f_X:D_X\to [0,1],\ x\mapsto \P(\{X=x\})\] &        
$X$ absolut stetig, wenn $\P_X$ Dichte $f_X$ besitzt\\
\multicolumn{2}{c}{\textit{Verteilungsfunktion}}  \\
$\F_X(x)=\sum\limits_{y\in D_X\cap (-\infty, x]} \P(\{X=y\})$ &
$\F_X(x)=\int\limits_{(-\infty,x]}f_X(y) d\lambda(y)$ \\
\multicolumn{2}{c}{\textit{Verteilung}}  \\
$\P_X(A)=\sum\limits_{x\in D_X\cap A} \P(\{X=x\})$ &
$\P_X(A)=\int\limits_{A} f_X(x) d\lambda(x)$ \\
        &               
\end{tabular}
\end{table}




\section{Verteilungen}

\textbf{Bernoulli-Verteilung} $X\sim\mathbf{B}(1,p)$ mit $p\in [0,1]$
\begin{itemize}
\item $\P(\{X=1\})=p,\ \P(\{X=0\})=1-p$

\item diskret mit $D_X=\begin{cases}
\{0,1\} 	& \text{falls } p\in (0,1)	\\
\{0\}	& \text{falls } p=0			\\
\{1\} 	& \text{falls } p=1
\end{cases}$

\item $\F_X(x)=\begin{cases}
0 	& \text{für } x<0				\\
1-p	& \text{für } 0\leq x <1			\\
1 	& \text{für } x\geq 1
\end{cases}$

\item $\E(X)=p$, $\Var(X)=p-p^2=p(1-p)$

\end{itemize}

\textbf{Binomial-Verteilung} $X\sim\mathbf{B}(n,p)$ mit $p\in [0,1]$
\begin{itemize}
\item $\forall k\in\{0,\ldots,n\}:\P(\{X=k\})=\binom{n}{k}\cdot p^k\cdot (1-p)^{n-k}$

\item diskret mit
$D_X=\begin{cases}
\{0,\ldots,n\} 	& \text{falls } p\in (0,1)	\\
\{0\}			& \text{falls } p=0			\\
\{n\} 			& \text{falls } p=1
\end{cases}$

\item $\E(X)=n\cdot p$, $\Var(X)=n\cdot p\cdot (1-p)$
\end{itemize}

\textbf{Hypergeometrische Verteilung} $X\sim\mathbf{H}(N,N_0,n)$
\begin{itemize}
\item $\forall l\in D_X:\P(\{X=l\})
=\frac{\binom{N_0}{l}\cdot\binom{N-N_0}{n-l}}{\binom{N}{n}}$

\item diskret mit 
$D_X=\{\operatorname{max}(0,n-(N-N_0)),\ldots,\operatorname{min}(N_0,n)\}$
\end{itemize}

\textbf{Poisson-Verteilung} $X\sim\mathbf{P}(\lambda)$ für $\lambda>0$
\begin{itemize}
\item $\forall k\in\N_0:\P(\{X=k\})=\exp(-\lambda)\cdot\frac{\lambda^k}{k!}$

\item diskret mit $D_X=\N_0$

\item $\E(X)=\lambda=\Var(X)$, $\E(X^2)=\lambda^2+\lambda$
\end{itemize}

\textbf{Geometrische Verteilung} $X\sim\mathbf{G}(p)$ mit $p\in(0,1]$
\begin{itemize}
\item $\forall k\in\N:\P(\{X=k\})=p\cdot(1-p)^{k-1}$

\item diskret mit $D_X=\N$

\item $\E(X)=\frac{1}{p}$, $\Var(X)=\frac{1-p}{p^2}$, $\E(X^2)=\frac{2-p}{p^2}$

\item Gedächtnislos: $\forall k_1,k_2\in\N \text{ mit } k_1<k_2:
\P(\{X>k_2\}\mid \{ X>k_2\}) = \P(\{X>k_2-k_1\})$\\
\end{itemize}

\textbf{Gleichverteilung} $X\sim\mathbf{U}([a,b])$ mit $-\infty<a<b<\infty$
\begin{itemize}
\item absstetig mit 
$f_X(x)=\begin{cases}
\frac{1}{b-a} 	& \text{falls } x\in [a,b]	\\
0				& \text{sonst }		
\end{cases}$

\item
$\F_X(x)=\begin{cases}
0 				& \text{falls } x<a			\\
\frac{x-a}{b-a}	& \text{falls } x\in [a,b]	\\
1				& \text{sonst}
\end{cases}$

\item $\E(X)=\frac{a+b}{2}$, $\Var(X)=\frac{(b-a)^2}{12}$,
$\E(X^2)=\frac{b^3-a^3}{3(b-a)}$
\end{itemize}

\textbf{Einpunktverteilung} $X\sim\mathbf{U}(\{c\})$ für ein $c\in\R$
\begin{itemize}
\item $P(\{X=c\})=c$

\item diskret mit $D_X=\{c\}$

\item $\F_X(x)=\begin{cases}
0 	&\text{falls } x<c			\\
1	& \text{falls } x\geq c
\end{cases}$

\item $\E(X)=c$, $\Var(X)=0$
\end{itemize}

\textbf{Exponentialverteilung} $X\sim\mathbf{Exp}(\lambda)$ für $\lambda>0$
\begin{itemize}
\item absstetig mit
$f_X:\R\to [0,\infty],\ x\mapsto\begin{cases}
\lambda\cdot\exp(-\lambda x) 	& \text{falls } x\geq 0	\\
0							& \text{sonst }		
\end{cases}$

\item $\F_X(x)=1-\exp(-\lambda x)$

\item $\E(X)=\frac{1}{\lambda}$, $\Var(X)=\frac{1}{\lambda^2}$,
$\E(X^2)=\frac{2}{\lambda^2}$

\item Gedächtnislos: $\forall s,t>0:\P(\{X>t+s\}\mid\{X>t\})=\P(\{X>s\})$\\
\end{itemize}

\textbf{Standard-Normalverteilung} $X\sim\mathbf{N}(0,1)$
\begin{itemize}
\item absstetig mit $f_X:\R\to[0,\infty],\
x\mapsto\frac{1}{\sqrt{2\pi\sigma^2}}\cdot\exp(-\frac{(x-\mu)^2)}{2\sigma^2})$

\item $\Phi(x):=\F_X(x)=
\frac{1}{\sqrt{2\pi}}\int_{-\infty}^{x}\exp(-\frac{y^2}{2})d\lambda(y)$
(keine explizite Formel)

\item $\forall z\in\R:\Phi(z)+\Phi(-z)=1$ \U

\item $\E(X)=0$, $\Var(X)=1$

\end{itemize}

\textbf{Normalverteilung} $X\sim\mathbf{N}(\mu,\sigma^2)$ für 
$\mu\in\R,\ \sigma>0$
\begin{itemize}
\item absstetig mit $f_X:\R\to[0,\infty],\ 
x\mapsto\frac{1}{\sqrt{2\pi\sigma^2}}\cdot\exp(-\frac{(x-\mu)^2)}{2\sigma^2})$

\item $X\sim\mathbf{N}(\mu,\sigma^2)
\Rightarrow a\cdot X+b\sim\mathbf{N}(a\cdot\mu+b,a^2\cdot\sigma^2)$

\item Also: $\frac{X-\mu}{\sigma}\sim\mathbf{N}(0,1)$

\item $\E(X)=\mu$, $\Var(X)=\sigma^2$
\end{itemize}

\textbf{Gleichverteilung} (mehrdimensional) $X\sim\mathbf{U}(S)$ für $S\subset\R^n$
\underline{endlich}
\begin{itemize}
\item $\forall x\in S: \P(\{X=x\})=\frac{1}{|S|}$

\textbf{Gleichverteilung} (mehrdimensional) $X\sim\mathbf{U}(G)$ für $G\in\B{d}$
\begin{itemize}
\item absstetig mit $f_X=\frac{1}{\lambda_d(G)}\cdot \1_G$

\item $\P_X(A)=\frac{\lambda_d(A\cap G)}{\lambda_d(G)}=
\int_A \frac{1}{\lambda_d(G)\cdot 1_G(x)}d\lambda(x)$
\end{itemize}

\item diskret mit $D_X=S$
\end{itemize}

\textbf{Standard-Normalverteilung} (mehrdimensional)
\begin{itemize}
\item absolutstetig mit 
$f_X:\R^d\to [0,\infty),\ 
x\mapsto(2\pi)^{-d/2}\cdot\exp(-\frac{1}{2}\sum_{i=1}^{d}x_i^2)$

\item $X=(X_1,\ldots,X_d)$ ist standard-normalverteilt 
$\Leftrightarrow$ $X_1,\ldots,X_d\ \operatorname{i.i.d}\sim\mathbf{N}(0,1)$
\end{itemize}

%Einpunkt-Verteilung

%\include{sections/04-erwart_var}
%\include{sections/05-grenzwert}
%\section{Sonstiges}


\textbf{Exponentialfunktion}:
\begin{itemize}
\item $\exp:\mathbb{C}\to\mathbb{C},\ x\mapsto\sum_{k=0}^\infty \frac{z^k}{k!}$

\item $\exp(a\cdot b)=\exp(a)+\exp(b)$

\item $\exp$ stetig, $\exp>0$

\item $\exp'=\exp$
\end{itemize}

\textbf{Natürlicher Logarithmus}
\begin{itemize}
\item $\ln: (0,\infty)\to\R:\ x\mapsto\exp^{-1}(x)$

\item $\ln(a\cdot b)=\ln(a)+\ln(b)$

\item $\ln(x)=\frac{1}{x}$
\end{itemize}

\textbf{Trigonometrische Funktionen}
\begin{itemize}
\item $\cos:\R\to\R:\ x\mapsto\sum_{k=0}^\infty (-1)^k\cdot\frac{x^{2k}}{(2k)!}$

\item $\sin:\R\to\R:\ x\mapsto\sum_{k=0}^\infty (-1)^k\cdot\frac{x^{2k+1}}{(2k+1)!}$

\item $\cos' = -\sin$, $\sin' = \cos$
\end{itemize}

\textbf{Reihen}
\begin{itemize}
\item \textbf{Harmonische Reihe}: $\sum_{n=1}^\infty \frac{1}{n}=\infty$

\item \textbf{Geometrische Reihe}: $\sum_{n=0}^\infty a\cdot x^n = \frac{a}{1-x}$
für $a,x\in\R$ mit $|x|<1$, divergent für alle anderen $x\in\R$

\item \textbf{Alternierende harmonische Reihe}:
$\sum_{n=1}^\infty (-1)^n\cdot\frac{1}{n}=-\ln(2)$
\end{itemize}

\end{document}
