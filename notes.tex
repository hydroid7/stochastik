\documentclass[11pt,a4paper,ngerman]{article}

% IMPORTS
\usepackage[ngerman]{babel}
\usepackage[utf8x]{inputenc}
\usepackage[colorinlistoftodos,prependcaption,textsize=tiny]{todonotes} % To-Dos
\usepackage{parskip} 	% Absätze ohne Einrückungen
\usepackage{mathtools} 	% Für Pfeil mit Erklärung
\usepackage{amsfonts} 	% Buchstaben mit Doppelstrich
\usepackage{amsmath}
\usepackage{amssymb}
\usepackage{amsthm}		
\usepackage{bbm}			% Indikator Funktion
\usepackage{verbatim}	% comment-env
\usepackage{wasysym}		% \lightning
\usepackage{graphicx}	% Einbinden von Grafiken
\usepackage{url}
\usepackage{hyperref}

% META
\date{07.02.2018}
\author{Maximilian Reif}
\title{Notizen zur Einführung in die Stochastik}

%--------------------------------------------------------------------------------------------------------------

\begin{document}

\begin{titlepage}
    \ \newline\newline\newline\newline\newline
	
	\begin{center}

		\huge Notizen zu\\
		\Huge\textbf{Einführung in die Stochastik}\\
		\huge bei Prof. Kaiser im WS 17/17\\
		\normalsize

		\vspace{1cm}
		\begin{tabular}[b]{l|l}
			\textbf{author} 		& Maximilian Reif
			\texttt{\href{mailto:reifmaxi@fim.uni-passau.de}
			{<reifmaxi@fim.uni-passau.de>}}\\
			\textbf{last change}	& \today \\
			\textbf{version} 	& 0.0.1\\
			\textbf{github} 		& \url{https://github.com/lordreif/stochastik}
		\end{tabular}
		\vspace{1cm}
		
	\end{center}
	
	\begin{figure}[b]
	\centering
	\includegraphics[width=0.6\textwidth]{img/fim.png}
	\end{figure}
	
\end{titlepage}

%\include{sections/00-annotation}

\newpage
%\tableofcontents\thispagestyle{empty}
\newpage

\setcounter{page}{1}
\include{sections/01-basics}
%\include{sections/02-zufallsvariablen}
%\include{sections/03-erwart_var}
%\include{sections/04-grenzwert}

\end{document}